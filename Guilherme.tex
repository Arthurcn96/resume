% "ModernCV" CV and Cover Letter
% LaTeX Template
% Version 1.1 (9/12/12)
%
% This template has been downloaded from:
% http://www.LaTeXTemplates.com
%
% Original author:
% Xavier Danaux (xdanaux@gmail.com)
%
% License:
% CC BY-NC-SA 3.0 (http://creativecommons.org/licenses/by-nc-sa/3.0/)
%
% Important note:
% This template requires the moderncv.cls and .sty files to be in the same 
% directory as this .tex file. These files provide the resume style and themes 
% used for structuring the document.

\hyphenation{Campinas}
\hyphenation{UNICAMP}


\documentclass[10pt, a4paper, roman]{moderncv} % Font sizes: 10, 11, or 12; paper sizes: a4paper, letterpaper, a5paper, legalpaper, executivepaper or landscape; font families: sans or roman


\usepackage[portuguese]{babel}
\usepackage[utf8]{inputenc}
\usepackage[T1]{fontenc}


\moderncvstyle{casual} % CV theme - options include: 'casual' (default), 'classic', 'oldstyle' and 'banking'

\moderncvcolor{purple} % CV color - options include: 'blue' (default), 'orange', 'green', 'red', 'purple', 'grey' and 'black'

%\usepackage{lipsum} % Used for inserting dummy 'Lorem ipsum' text into the template

\usepackage[scale=0.80]{geometry} % Reduce document margins
%\setlength{\hintscolumnwidth}{3cm} % Uncomment to change the width of the dates column
\setlength{\makecvtitlenamewidth}{10cm} % For the 'classic' style, uncomment to adjust the width of the space allocated to your name

\firstname{Guilherme} 
\familyname{Lucas}

% All information in this block is optional, comment out any lines you don't need
\title{Curriculum Vitae}

\mobile{+55 19 98279-1826}

\email{guilherme.slucas@gmail.com}

\begin{document}

\makecvtitle % Prints the CV title

\section{Experiência}

\subsection{Profissional}

\cventry{2017 -- Presente}{Estágio em Data Science}{CETAX}{São Paulo}{Brasil}
{
    Desenvolvimento de aplicações envolvendo ciência de dados e big data, desde o design de API's para obtenção de dados
    até a criação de modelos de Aprendizado de Máquina.
    Ajudei também a implementar a cultura de code review.\\
    Competências envolvidas: Python, Spark e Sistemas GNU/Linux.
}

\subsection{Acadêmica}
\cventry{2016 -- 2017}{Iniciação Científica}{Unicamp}{Campinas}{Brasil}
{
    Desenvolvimento de aplicações para diferentes áreas, além de cuidar da infraestrutura e escrever textos e tutoriais sobre o que foi 
    feito, lidos por pessoas de diversos países.\\ 
    Competências: Sistemas GNU/Linux, C/C++, Shell Script, Python, Docker, Jenkins e Ansible.\\ 
    Projeto em parceria IBM, Petrobras (CEPETRO) e Unicamp.\\
    Orientador: Rodolfo Jardim de Azevedo.
}


\subsection{Voluntária}
\cventry{2017 -- Presente}{Membro}{Livrecamp - Unicamp}{Campinas}{Brasil}
{
    Membro do LivreCamp, grupo interessado em difundir software e conhecimento livres. Atuação
    como monitor e instrutor em cursos de Shell Script e GNU/Linux básico.
}
\cventry{2016 -- 2016}{Monitoria}{Programa de Auxílio Discente}{Campinas}{Brasil}
{
    Monitoria da matéria Circuitos Lógicos (MC602). As atividades foram atendimentos para sanar dúvidas 
    dos alunos, correção de listas de exercícios e manutenção dos softwares usados.
}
\cventry{2015 -- 2016}{Representante Discente}{Instituto de Computação}{Campinas}{Brasil}
{
Representação dos alunos Computação Unicamp em questões sobre a 
    infraestrutura do Instituto.
}

\cventry{2014 -- 2015}{Comissão Organizadora}{Semana de Computação da Unicamp}{Campinas}{Brasil}
{
Membro da equipe de infraestrutura da Semana de Computação da Unicamp, entrando em contato com palestrantes e planejando atividades.
}

\cventry{2014 -- 2016}{Coordenador Administrativo}{CACo - Centro Acadêmico da Computação}{Campinas}{Brasil}
{
Representação dos alunos dos cursos de Engenharia e Ciência da Computação.
}

\section{Educação}

\cventry{2014 -- Presente}{Engenharia de Computação}{Universidade Estadual de Campinas (UNICAMP)}{Campinas}{Brasil}{}

%\cventry{2009 -- 2011}{Ensino Médio}{Colégio Objetivo de Pirassununga}{Pirassununga}{Brasil}
%{
%Participação em várias edições de Olimpíadas Científicas, como Matemática, Física e Astronomia.}

\section{Línguas}

\cvitemwithcomment{Nativo}{Português}{}
\cvitemwithcomment{Fluente}{Inglês}{}
\cvitemwithcomment{Básico}{Espanhol}{}
\cvitemwithcomment{Básico}{Francês}{}
\section{Conhecimento Técnico}

\cvitem{Intermediário}{Java, Django, \LaTeX, HTML, CSS, Docker, Jenkins, Ansible, SQL, Hadoop, Spark, Go}
\cvitem{Bom}{Python, C/C++, Git, Sistemas Operacionais GNU/Linux, Shell Script, Lisp, Haskell}

\end{document}
