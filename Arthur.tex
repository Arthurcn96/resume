% "ModernCV" CV and Cover Letter
% LaTeX Template
% Version 1.1 (9/12/12)
%
% This template has been downloaded from:
% http://www.LaTeXTemplates.com
%
% Original author:
% Xavier Danaux (xdanaux@gmail.com)
%
% License:
% CC BY-NC-SA 3.0 (http://creativecommons.org/licenses/by-nc-sa/3.0/)
%
% Important note:
% This template requires the moderncv.cls and .sty files to be in the same 
% directory as this .tex file. These files provide the resume style and themes 
% used for structuring the document.


\documentclass[10pt, a4paper, roman]{moderncv} % Font sizes: 10, 11, or 12; paper sizes: a4paper, letterpaper, a5paper, legalpaper, executivepaper or landscape; font families: sans or roman

\usepackage[portuguese]{babel}
\usepackage[utf8]{inputenc}
\usepackage[T1]{fontenc}

\moderncvstyle{casual} % CV theme - options include: 'casual' (default), 'classic', 'oldstyle' and 'banking'

\moderncvcolor{blue} % CV color - options include: 'blue' (default), 'orange', 'green', 'red', 'purple', 'grey' and 'black'

%\usepackage{lipsum} % Used for inserting dummy 'Lorem ipsum' text into the template

\usepackage[scale=0.90]{geometry} % Reduce document margins
%\setlength{\hintscolumnwidth}{3cm} % Uncomment to change the width of the dates column
\setlength{\makecvtitlenamewidth}{10cm} % For the 'classic' style, uncomment to adjust the width of the space allocated to your name

\usepackage[hidelinks]{hyperref}
\usepackage{xcolor}

\hypersetup{
    colorlinks=true,
    linkcolor=blue,
    urlcolor=blue
}

\firstname{Arthur}
\familyname{Novais}

% All information in this block is optional, comment out any lines you don't need
\title{Curriculum Vitae}

\mobile{+55 27 99915-6801}
\email{Arthurcn96@gmail.com}

\begin{document}
\makecvtitle % Prints the CV title

{\LARGE \hspace{2.15cm}\textbf{Arthur Costa de Novais} \\  \\}
\vspace{.3cm}
{\Large\hspace{2cm} \textbf{Computer Vision | Machine Learning | Data Science} \\  \\}
\vspace{0.2cm}
\hspace{2cm} 
\textbf{\href{mailto:arthurcn96@gmail.com}{Arthurcn96@gmail.com}} | 
\textbf{\href{https://www.linkedin.com/in/arthur-novais-201420}{LinkedIn} }

\vspace{0.2cm}

\section{Resumo}
\vspace{0.2cm}

\cvitem{}{Profissional com graduação em \textbf{Ciência da Computação}. Experiência sólida no desenvolvimento de soluções de \textbf{visão computacional} e \textbf{aprendizado profundo}, utilizando bibliotecas como \textbf{TensorFlow}, \textbf{PyTorch} e \textbf{OpenCV}. Atuação destacada em projetos de classificação e detecção de objetos em tempo real, com integração de soluções embarcadas e aplicações industriais. Familiaridade com ambientes Linux, Docker e Git para gestão de projetos e implantação de modelos.}
\vspace{0.5cm}

\section{Experiência}

\subsection{Profissional}

\vspace{0.2cm}
\cventry{2022 - 2024}{Computer Vision Engineer}{Vixteam Consultoria e Sistemas}{Espirito Santo}{Brasil}
{
    Atuando como engenheiro de visão computacional em projetos industriais de alta complexidade, combinando aprendizado profundo, sistemas embarcados e infraestrutura moderna.\\
    \begin{itemize}
      \item Desenvolvimento soluções de visão computacional para a ArcelorMittal utilizando Python, TensorFlow e OpenCV;
      \item Implementação de arquiteturas de redes neurais convolucionais (CNNs) para detecção de defeitos na produção de aço;
      \item Integração de modelos treinados com sistemas embarcados para análise em tempo real.
    \end{itemize}
}

\vspace{0.4cm}
\cventry{2021 - 2022}{Software Engineer}{Vixteam Consultoria e Sistemas}{Espirito Santo}{Brasil}
{
    Contribuiu para o desenvolvimento de ferramentas robustas em ambientes corporativos, integrando linguagens de programação versáteis e sistemas legado.\\
    \begin{itemize}
      \item Desenvolveu soluções de software utilizando C++, Java e SQL;
      \item Criou ferramentas customizadas e deu suporte a processos existentes, utilizando Git para controle de versão.
    \end{itemize}
}



\vspace{0.4cm}
\cventry{2021}{Estágio}{Vixteam Consultoria e Sistemas}{Espirito Santo}{Brasil}
{
    Contribuiu para o desenvolvimento de ferramentas robustas em ambientes corporativos, integrando linguagens de programação versáteis e sistemas legado.\\
    \begin{itemize}
      \item Desenvolveu soluções de software utilizando C++, Java e SQL;
      \item Criou ferramentas customizadas e deu suporte a processos existentes, utilizando Git para controle de versão.
    \end{itemize}
}

\vspace{0.6cm}
\subsection{Acadêmica}

\vspace{0.2cm}
\cventry{2022}{Graduação}{UFES}{Espirito Santo}{Brasil}
{
  Graduado na Universidade Federal do Espírito Santo com um foco em Visão Computacional com  projetos e TCC em focados em classificação e detecção.
}

\vspace{0.2cm}

\vspace{0.4cm}

\cventry{2019}{Project Manager}{Adapti - Soluções Web}{Espirito Santo}{Brasil}
{
    Coordenou equipes multidisciplinares para entrega eficiente de projetos web, aplicando boas práticas de gestão e desenvolvimento ágil\\
    \begin{itemize}
      \item Planejou e gerenciou projetos de sistemas web utilizando a metodologia ágil SCRUM;
      \item Criou ferramentas customizadas e deu suporte a processos existentes, utilizando Git para controle de versão.
    \end{itemize}
}

\vspace{0.4cm}
\cventry{2018 - 2019}{Software Engineer}{Adapti - Soluções Web}{Espirito Santo}{Brasil}
{
    Criou soluções web modernas, aplicando conhecimento em tecnologias front-end e back-end para atender às demandas de clientes.\\
    \begin{itemize}
      \item Desenvolveu aplicações web com HTML5, CSS3, JavaScript e frameworks como Bootstrap e Laravel;
      \item Implementou sistemas seguindo o padrão de projeto MVC, utilizando Git para controle de versão.
    \end{itemize}
}





\vspace{0.2cm}
\cventry{2019 - 2020}{Projeto de Iniciação Científica}{UFES}{Espirito Santo}{Brasil}
{
	\begin{itemize}
		\item \textbf{Título do Projeto}: Mecanismos de Compreensão e Interpretação da Cor na Visão Computacional.
		\item \textbf{Título do Subprojeto}: Proposta de uma Nova Estrutura de Filtragem nas Redes Neurais Convolucionais.
	\end{itemize}
  No projeto percebi a importância de uma boa \textbf{organização} e distribuição de tarefas, e como é valioso os contatos e amizades que são formados.
}


\vspace{0.2cm}


\subsection{Voluntária}

\vspace{0.2cm}
\cventry{2019 - 2020}{Voluntários Independentes Pelo Amigo(VIPA)}{ONG}{Espirito Santo}{Brasil}{Trabalho volutário em uma Organização que trata de cachorros abandonados, ajudando no tratamento, organização e limpeza do recinto.}

\vspace{0.5cm}
\section{Cursos}

\cvitem{}{\href{https://arthurcn96.github.io/certificates.html}{Fundamentals of Lakehouse Architecture}}
\cvitem{}{\href{https://arthurcn96.github.io/certificates.html}{Fundamentals of Delta Lake}}
\cvitem{}{\href{https://arthurcn96.github.io/certificates.html}{Fundamentals of the Databricks Lakehouse Platform}}
\cvitem{}{\href{https://arthurcn96.github.io/certificates.html}{Remote Monitoring and Diagnostics (RM\&D)}}

\vspace{0.5cm}
\section{Línguas}

\vspace{0.2cm}

\cvitemwithcomment{Nativo}{Português}{}
\cvitemwithcomment{Fluente}{Inglês}{}
\cvitemwithcomment{Básico}{Francês}{}
\vspace{0.5cm}
\section{Conhecimento Técnico}

\cvitem{}{\textbf{Machine Learning e Deep Learning}: Experiência prática no desenvolvimento de modelos utilizando TensorFlow, PyTorch e Scikit-learn.}
\cvitem{}{\textbf{Visão Computacional}: Visão Computacional: Domínio de bibliotecas como OpenCV e Imutils para construção de pipelines de análise de imagens.}
\cvitem{}{\textbf{Infraestrutura e DevOps}: Conhecimento com Docker, Linux e Git para criação e gerenciamento de ambientes de desenvolvimento.}
\cvitem{}{\textbf{Linguagens de Programação}: Proficiência em Python, com experiência adicional em C++ e Java.}


\end{document}
